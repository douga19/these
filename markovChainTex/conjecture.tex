\documentclass[11pt]{article}
% \documentclass[twocolumn]{article}

\usepackage[francais]{babel}
\usepackage[utf8]{inputenc}
\usepackage{enumerate}
\usepackage{amssymb}
\usepackage{amsfonts}
\usepackage{amsthm}
\usepackage{graphicx}
\usepackage[ruled,vlined,french]{algorithm2e}

% New commands
\renewcommand\qedsymbol{$\blacksquare$}
% \renewcommand\diffsim{$x \bigtriangleup y$}

% New Environment
\newtheorem{conjecture}{Conjecture}
\newtheorem{proposition}{Proposition}

\begin{document}

\title{Différence symétrique et distance dans ($X_t$)}
\author{Julien David, Lionel Pournin, Rakotonarivo Rado}
\maketitle

\section{Définition}

Soient deux polytopes $x = \{u_1, \dots , u_l\}$ et $y = \{v_1, \dots , v_k\} \in \Omega$. On définit par $x \bigtriangleup y$ la \textbf{différence symétrique} entre $x$ et $y$ telle que
\begin{equation}
  x \bigtriangleup y = \{ u \in x : u \notin y \ \mbox{et} \ v \in y : v \notin x \}
\end{equation}

On peut voir la différence symétrique de manière ensembliste comme étant $x \bigtriangleup y = x \cup y \setminus x \cap y$ cependant quelques précisions sont à mentionner:

\begin{itemize}
  \item $x \cup y$ ne constitue pas forcément une enveloppe convexe.
  \item $|x \cup y| = |x| + |y|$ si $x \cap y = \emptyset$.
  \item $x \bigtriangleup y$ est maximal quand $x$ et $y$ n'ont aucun sommet en commun.
\end{itemize}

\begin{proposition}
  La distance entre $x$ et $y$ dans le graphe de $X_t$ est bornée par le cardinal de $x \bigtriangleup y$, on notera cette distance $\delta(x,y)$ et on a:
  \begin{equation}
    \delta(x,y) \geq{|x \bigtriangleup y|}
  \end{equation}
\end{proposition}

\begin{proof}
  Considérons $x$ et $y \in \Omega$. Comme $x \bigtriangleup y$ constitue l'ensemble des sommets sur lesquels $x$ diffère de $y$ et réciproquement, passer de $x$ en $y$ avec un nombre minimal d'étapes consiste à choisir un chemin qui fera en sorte de réduire $x \bigtriangleup y$ d'un sommet à chaque étape. Par conséquent, il faut au moins $|x \bigtriangleup y|$ étapes pour passer de $x$ en $y$.
\end{proof}

% Pour tout suite d'étapes intermédiaires $(z_i)$ avec $1\leq{i}\leq{\delta(x,y)}$ et $z_i \in \Omega$ on a:
%
% \begin{equation}
%   |x \bigtriangleup y| \geq{|z_i \bigtriangleup y|}
% \end{equation}

Mettre en place la notion de différence symétrique entre deux états $x$ et $y$ va permettre d'assurer l'irréductibilité de notre chaîne $(X_t)$. En effet passer de $x$ en $y$ consiste en à trouver un nombre fini d'opérations d'ajouts et de suppressions de sommets. L'idéal serait de directement ajouter des sommets de $y$ et de supprimer ceux de $x$. Toutefois on peut tomber dans des cas où on ne peut ni supprimer des sommets de $x$ (le cas où $x$ est un simplexe) ni ajouter des sommets de $y$. On met alors en emphase plusieurs cas à distinguer:

\begin{enumerate}
  \item $x$ est n'est pas un simplexe.
  \begin{enumerate}
    \item $x \subset y$: On ajoute un élément de $y \setminus x$
    \item $x \not\subset y$: On supprime un élément de $x \setminus y$
  \end{enumerate}
  \item $x$ est un simplexe.
  \begin{enumerate}
    \item Si on peut ajouter un élément de $y \setminus x$ alors on le fait
    \item Sinon:
    \begin{enumerate}
      \item Ajouter un point extérieur à $x \bigtriangleup y$
      \item Supprimer un élémént de $x \setminus y$
      \item Ajouter un élément de $y \setminus x$
    \end{enumerate}
  \end{enumerate}
\end{enumerate}

\begin{proposition}
  On pose la conjecture suivante: $\exists z \in \Omega$, tel que $x \bigtriangleup y \supset z \bigtriangleup y$, pour lequel on a $\delta(x,z) \leq{3}$
\end{proposition}

Cette conjecture nous dit qu'on peut trouver un état transitoire $z$ entre $x$ et $y$ tel qu'en au plus de 3 étapes on peut réduire $x \bigtriangleup y$ d'un sommet.


\end{document}
