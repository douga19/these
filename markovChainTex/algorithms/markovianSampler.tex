\SetKwInput{Entree}{Entrées}
\SetKwInput{Sortie}{Sortie}
\SetKwIF{Si}{SinonSi}{Sinon}{si}{alors}{sinon si}{alors}{finsi}
\SetKwFor{Tq}{tant que}{faire}{fintantque}
\SetKw{Renvoie}{renvoyer}

\begin{algorithm}[H]\label{algo1}
  \LinesNumbered
  \DontPrintSemicolon
  \Entree{la dimension $d$, le côté $k$ de l'hypercube}
  \Sortie{$(d,k)$-polytope tiré aléatoirement dans $[0,k]^d$}
  \BlankLine

  état initial arbitraire $x_0$\;
  \Tq{on n'a pas atteint la distribution stationnaire}{
  tirer aléatoirement un point $v$ dans $[0,k]^d$\;
  \uSi{$v \in x_0$ et $|x_0| > d + 1$}{
    $x_0$ = $x_0 - \{ v\}$\;
   }
  \SinonSi{$|Conv(x_0 \cup \{ v\})|$ == $|x_0| + 1$}{
    $x_0$ = $x_0 \cup \{ v\}$\;
  }
 }
 \Renvoie{$x_0$}
 \caption{$\Gamma(d,k)$: Engendrer aléatoirement un $(d,k)$-polytope}
\end{algorithm}
