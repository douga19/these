\documentclass[twocolumn]{article}
%\documentclass[11pt]{article}
\usepackage[utf8]{inputenc}
\usepackage{enumerate}
\usepackage{amssymb}
\usepackage{amsfonts}

\begin{document}

%titre
\title{Idées de preuve que les (\textit{d},2)-polytopes sont Hirsch}
\author{Rado Rakotonarivo}
\maketitle

\section{Motivations}
L'étude du diamètre du graphe des polytopes a connu un engouement assez conséquent au cours de ces dernières décennies \cite{deza2016improved}, \cite{naddef1989hirsch} et \cite{del2016diameter}. Ce diamètre constitue une borne inférieure de la complexité de l'algorithme de simplexe, lors de la résolution d'un problème d'optimisation. Effectivement, résoudre un programme linéaire se réduit à parcourir le graphe d'un polytope dont la dimension correspond au nombre de variables et le nombre de facettes au nombre de contraintes.

Toujours dans cette même optique, Hirsch émmet une conjecture  mettant en relation le diamètre $\delta$, le nombre de facettes $n$ ainsi que la dimension $d$ d'un polytope suivant l'inégalité:
$$
	n-d\geq{\delta}
$$
Cette conjecture a été prouvée pour certains cas particuliers mais néanmoins reste à démontrer dans le cas général des polytopes. Pour les
(\textit{d,k})-polytopes, i.e les polytopes entiers contenu dans l'hypercube $[0,k]^d$, Naddef montre de façon élémentaire que pour $k=1$ la conjecture est vraie \cite{naddef1989hirsch}.

Nous allons nous intéresser au cas où $k$ vaut 2. Dans ce rapport, l'idée est de prouver Hirsch pour le cas $k = 2$, on va se baser sur les résultats antérieurs de Naddef ainsi que de Del Pia et Michini.

Pour rappel, Del Pia et Michini arrivent à prouver que pour $k\geq{2}$, on a:
$$
	\delta\leq{\lfloor{(k-\frac{1}{2})}d\rfloor}
$$
Et que pour $k=2$, $\delta$ est au plus $\lfloor{\frac{3}{2}}d\rfloor$. De plus, quelque soit $d$, il existe un (\textit{d,2})-polytope qui atteint cette borne.

Ce ne sont que des éléments sur lesquels notre raisonnement va partir et il se peut que l'on aboutisse pas à notre objectif; par contre c'est une bonne manière pour commencer notre travail dans la manière où d'autres propriétés intéressantes sur le diamètre des (\textit{d,k})-polytopes pourraient en découler.

\section{Idées de preuves}
L'idée est de raisonner par récurrence sur la dimension. C'est d'ailleurs la méthode de démonstration utilisée dans\cite{naddef1989hirsch} et \cite{del2016diameter}. Le procédé sera comme suit:
\begin{itemize}
\item Supposons que $n-d\geq{\delta(P)}$ soit vraie pour  $d\leq{d^*}$.
\item Montrons ensuite que $n-d\geq{\delta(P)}$ pour $d = d^* + 1$
\end{itemize}

La vérification des cas de base étant triviale, il nous faut distinguer deux cas: un cas où $\delta(P)\leq{d}$ et le cas où $\delta(P)>d$. Cette distinction suit respectivement les résultats de Naddef et de Del Pia Michini.

\subsection{Premier cas: $\delta(P)\leq{d}$}
Soit $P$ notre (\textit{d,2})-polytope.
Pour $u$ et $v$ deux sommets de $P$, notons par $d_P(u,v)$ la distance au sens du graphe de $u$ et de $v$ dans $P$. Prenons maintenant $u$ et $v$ tel que $d_P(u,v) = \delta(P)$.

\begin{enumerate}[i]
\item Si $u$ et $v$ appartiennent à une même facette $F$ de $P$, on sait que: $dim(F)=d^*$ et $n(F)\leq{n(P)-1}$.

Appliquons ensuite l'hypothèse de récurrence pour $F$.
$$
	n(F)-dim(F)\geq{d_F(u,v)}\geq{d_p(u,v)}=\delta(P)
$$
et comme $n(P)-1-d^*\geq{n(F)-dim(F)}$, on a:
$$
	\begin{array}{c}
		n(P)-1-d^*\geq\delta(P)\\
		n(P)-1-d+1\geq\delta(P)\\
		n-d\geq\delta
	\end{array}
$$

\item Si $u$ et $v$ n'appartiennent à une même facette $F$ de $P$, on peut supposer dans ce cas que $F$ est une facette que contient $v$ mais qui ne contient pas $u$. On a toujours $dim(F)=d^*$ et $n(F)\leq{n(P)-1}$. Mais de plus, on sait que $u$ est au moins contenue dans $d^*+1$ facettes de $P$, il en est de même pour $v$; cette information nous donne une troisième équation: $n(P)\geq{2(d^*+1)}$.

L'hypothèse de récurrence sur $F$ donne:
$$
	n(F\geq{\delta(F)+d^*})
$$
Et comme $n(P)-1\geq{n(F)}$, on obtient:
$$
	\begin{array}{c}
		n(P)-1-\geq\delta(F)+d^*\\
		n(P)-(d^*+1)\geq\delta(F)\\
	\end{array}
$$
Or $n(P)\geq{2(d^*+1)}$ i.e:
$$
	n(P)-(d^*+1)\geq{2(d^*+1)}-(d^*+1)=d^*+1=d\geq{\delta(P)}
$$
On conclut que $n - d \geq{\delta}$
\end{enumerate}

\subsection{Deuxième cas: $\delta(P)>d$}
Dans cette section, on va débuter notre raisonnement sur la manière de faire dans \cite{del2016diameter}. Del Pia et Michini partent le principe de construction d'un chemin reliant deux sommets $u$ et $v$ de $P$ de telle sorte que ce chemin parcoure autant de nouvelle facettes que possibles pour vérifier Hirsch. Comme nous nous intéressons au cas où $k=2$, leur résultat nous ajoute une contrainte supplémentaire sur le diamètre i.e. $\delta(P)\leq{\lfloor{\frac{3}{2}}d\rfloor}$.

La principale piste pour prouver Hirsch pour $k=2$ réside dans leur manière de procéder. Considérer deux sommets $u$ et $v$ de $P$: si elles sont contenues dans une même facette $F$, on applique Naddef et on est content; sinon, trouver un chemin entre $u$ et $v$ de telle sorte que l'on découvre au fur et à mesure que l'on avance de $u$ vers $v$ un certain nombre de nouvelles facettes (mises à part les $2d$ qui nous sont donnés gratuitement du fait que $u$ et $v$ ne sont pas contenues dans une même facette $F$).

Leurs résultats offrent plusieurs propriétés utiles dans la suite de notre travail néanmoins il me reste à bien les utiliser pour essayer de prouver Hirsch pour $k=2$.

Voici quelques questions qui ont été soulevés et qui me reste encore à voir:

\begin{enumerate}[i]
	\item Est-ce que $d_P(u,v)\leq{d_F(u',v)+1}$ si au moins $\exists{u'}$ à distance 1 de $u$ qui tombe dans $F$?
	\item Pour $k=2$ quelle est la relation entre $\delta(P)$ et $\delta(F)$?
	\item Quelle sera la manière de construire le chemin pour notre relation de récurrence pour que Hirsch soit vérifiée i.e. combien de nouvelles facettes faudrait-il découvrir, $\delta(P)-d\leq{\lfloor{\frac{d}{2}}\rfloor}$?
\end{enumerate}

\section{Nouvelle section}

\clearpage

\begin{thebibliography}{9}
\bibitem{deza2016improved}
	Antoine Deza, Lionel Pournin.
	\emph{Improved bounds on the diameter of lattice polytopes}.
	arXiv preprint arXiv:1610.00341.
	2016.
\bibitem{naddef1989hirsch}
	Denis Naddef.
	\emph{The Hirsch conjecture is true for (0, 1)-polytopes}.
	Mathematical Programming, Springer.
	1989.
\bibitem{del2016diameter}
	Alberto Del Pia, Carla Michini.
	\emph{On the diameter of lattice polytopes}.
	Discrete \& Computational Geometry, Springer.
	2016.
\end{thebibliography}



\end{document}
